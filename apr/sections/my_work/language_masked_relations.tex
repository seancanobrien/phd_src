%! TEX root = ../../main.tex

\subsection{Using Languages to choose relations}

Let $F \colon \cset \to \cgrp$ be the free functor and denote its action on objects as $F_S \coloneqq F(S)$ and its action on morphisms as  $f_* \coloneqq F(f \colon S \to T)$.
To each quotient $G$ of $F_S$ we associate the surjective homomorphism $\pi_{(S,G)} \colon F_S \to G$ which is projection.
In this context, we can frame a basic group theoretic fact.

\begin{basic_fact}
	Suppose $G_1 \cong \GroupPres{S_1 \relations R_1}$ and $G_2 \cong \GroupPres{S_2}$.
	Given a map $f \colon S_1 \to S_2$, if there exists a map  $h$ such that the following diagram commutes, then $h$ is a homomorphism.
	\begin{equation}
		\begin{tikzcd}
			S_1 \ar[d, "f"] \ar[r, hook, "i_{S_1}"] & F_{S_1} \ar[d, "f_*"] \ar[rr, "\pi_{(S_1, G_1)}"] & & \ar[d, color=red, "h"] G_1 \\
			S_2 \ar[r, hook, "i_{S_2}"] & F_{S_2} \ar[rr, "\pi_{(S_2, G_2)}"] & & G_2
		\end{tikzcd}
		\label{eqn:homomorphism_squares}
	\end{equation}
	\label{thm:homomorphism_squares}
\end{basic_fact}

Now consider \eqref{eqn:homomorphism_squares}, but replace $F_{S_1}$ with some subset $Q \subseteq F_{S_1}$.
We will now construct a group, which we will denote $G^Q$, where such a diagram defines a homomorphism from $G^Q$ to $G_2$.

\begin{definition}[Group with relations visible in $Q$]
	Suppose we have a group $G \cong \GroupPres{S}$.
	Fix some $Q \subseteq F_S$ such that $S \subseteq Q$.
	Let $\pi \coloneq \pi_{(S,G)}$.
	We have the following maps.
	\[
		\begin{tikzcd}
			S \ar[r, hook, "i"] & Q \ar[r, "\pi"] & \pi(Q)
		\end{tikzcd}
		.\]
	Define the \emph{group $G$ with relations visible in $Q$}, to be
	\[
		G^Q \coloneqq \GroupPres{\pi(Q) \relations \Set{\pi(q) = (\pi \circ i)_*(q) \given q \in Q} }
		.\]
	\label{def:G_Q}
\end{definition}

Do not think of  the generators $\pi(Q)$ as elements of $G$.
They are abstract generators.
Specifically,  $G^Q$ is a quotient of  $F_{\pi(Q)}$.
Thus, our relations should be equations in $F_{\pi(Q)}$, which they are.
$G^Q$ is generated by $\pi(S) \subseteq \pi(Q)$.
$G$ is a quotient of  $G^Q$, which we will explore later.

\begin{lemma}
	Let $G \cong \GroupPres{S}$ be a group and consider the following setup.
	\[
		\begin{tikzcd}
			S \ar[r, hook, "i_S"] & F_S \ar[rr, "\pi_{(S,G)}"] & & G \ar[r, hook, "i_G"] & F_G \ar[rr, "\pi_{(G,G)}"] & & G
		\end{tikzcd}
	\]
	The following maps are equal.
	\[
		\pi_{(G,G)} \circ i_G \circ \pi_{(S,G)} = \pi_{(G,G)} \circ (\pi_{(S,G)} \circ i_S)_*
		.\]
	\label{lem:group_projections}
\end{lemma}

\begin{proof}
	Both maps are homomorphism which agree with $\pi_{\left( S,G\right)}$ on the generating set $S$.
\end{proof}

\begin{proposition}
	Suppose we have two groups $G_1 \cong \GroupPres{S_1}$ and $G_2 \cong \GroupPres{S_2}$.
	Fix some $Q \subseteq F_{S_1}$ such that  $S \subseteq Q$.
	Let $\overline{Q} \coloneqq \pi_{(S_1, G_1)}(Q) $.
	If in the following diagram \eqref{eqn:G_Q_homomorphism_setup}, if there exists a map $f$ that makes the diagram commute, then there is a homomorphism $h \colon (G_1)^Q \to G_2$.
	\begin{equation}
		\begin{tikzcd}
			S_1 \ar[d, "\theta"] \ar[r, hook, "i_{S_1}"] & Q \ar[d, "\theta_*"] \ar[rr, "\pi_{(S_1,G_1)}"] & & \overline{Q} \ar[d, color=red, "\exists f ?"] \\
			S_2 \ar[r, hook, "i_{S_2}"] & F_{S_2} \ar[rr, "\pi_{(S_2,G_2)}"] & & G_2
		\end{tikzcd}
		\label{eqn:G_Q_homomorphism_setup}
	\end{equation}
	\label{thm:G_Q_homomorphism}
\end{proposition}

I will include the proof only because I find it quite satisfying.

\begin{proof}
	Suppose we have such a map $f$.
	We use the following commutative diagram to define some inclusion maps and as a reference for the setup.
	\begin{equation}
		\begin{tikzcd}
			S_1 \ar[d, "\theta"] \ar[r, hook, "i_{S_1}"] & Q \ar[d, "\theta_*"] \ar[rr, "\pi_{(S_1, G_1)}"] & & \overline{Q} \ar[d, "f"] \ar[r, hook, "i_{\overline{Q}}"] & F_{\overline{Q}} \ar[d, "f_*"]
			\\ S_2 \ar[r, hook, "i_{S_2}"] & F_{S_2} \ar[rr, "\pi_{(S_2, G_2)}"] & & G_2 \ar[r, hook, "i_{G_2}"] & F_{G_2} \ar[rr, "\pi_{(G_2, G_2)}"] & & G_2
		\end{tikzcd}
		\label{eqn:G_Q_proof_setup}
	\end{equation}
	We can construct a homomorphism if $\pi_{(G_2, G_2)} \circ f_*(R) = \Set{1}$, where $R$ are the relations for $G^Q$, defined in \cref{def:G_Q}.
	For some $q \in Q$, the corresponding element in $R$ equalling the identity is  $i_{\overline{Q}} \circ \pi_{(S_1,G_1)}(q)(\pi_{(S_1,G_1)} \circ i_1)_*(q)^{-1}$.
	Using that $\pi_{(G_2,G_2)} \circ f_*$ is a homomorphism, we have
	\begin{align*}
		\pi_{(G_2,G_2)} \circ f_* \left(i_{\overline{Q}} \circ \pi_{(S_1,G_1)}(q)(\pi_{(S_1,G_1)} \circ i_1)_*(q)^{-1}\right) =
		\\ \left(\pi_{(G_2,G_2)} \circ f_* \circ i_{\overline{Q}} \circ \pi_{(S_1,G_1)} (q)\right) \left( \pi_{(G_2,G_2)} \circ f_* \circ\left( \pi_{(S_1,G_1)} \circ i_{S_1}\right)_*(q)^{-1} \right).
	\end{align*}
	We will concentrate on each factor separately.

	First we consider the first factor.
	Since $\pi_{(S_1, G_1)}(q) \in \overline{Q}$, by the rightmost commuting square of \eqref{eqn:G_Q_proof_setup}, we have $f_* \circ i_{\overline{Q}} \circ \pi_{(S_1, G_1)}(q) = i_{G_2} \circ f \circ \pi_{(S_1,G_1)} (q)$.
	This gives us
	\begin{align*}
		\pi_{(G_2,G_2)} \circ f_* \circ i_{\overline{Q}} \circ \pi_{(S_1,G_1)} (q) = \pi_{(G_2,G_2)} \circ i_{G_2} \circ f \circ \pi_{(S_1,G_1)}(q).
	\end{align*}
	We then use the middle commuting square of \eqref{eqn:G_Q_proof_setup} to give us
	\begin{align*}
		\pi_{(G_2,G_2)} \circ i_{G_2} \circ f \circ \pi_{(S_1,G_1)}(q) = \pi_{(G_2,G_2)} \circ i_{G_2} \circ \pi_{(S_2,G_2)} \circ \theta_*(q).
	\end{align*}
	We then use \cref{lem:group_projections} and functoriality, giving us
	\begin{align*}
		\pi_{(G_2,G_2)} \circ i_{G_2} \circ \pi_{(S_2,G_2)} \circ \theta_*(q) & = \pi_{(G_2,G_2)} \circ \left( \pi_{(S_2,G_2)} \circ i_{S_2} \right)_*  \circ \theta_*(q)
		\\ &= \pi_{(G_2,G_2)} \circ \left( \pi_{(S_2,G_2)} \circ i_{S_2} \circ \theta \right)_*(q)
	\end{align*}

	We now concentrate on the second factor.
	Using functoriality, then the middle commuting square of \eqref{eqn:G_Q_proof_setup}, we get
	\begin{align*}
		\pi_{(G_2,G_2)} \circ f_* \circ\left( \pi_{(S_1,G_1)} \circ i_{S_1}\right)_*(q)^{-1} & = \pi_{(G_2,G_2)} \circ \left( f \circ \pi_{(S_1,G_1)} \circ i_{S_1}\right)_*(q)^{-1}
		\\ &= \pi_{(G_2,G_2)} \circ \left(\pi_{(S_2,G_2)} \circ \theta_* \circ i_{S_1}\right)_*(q)^{-1}.
	\end{align*}
	Then, we use the leftmost commuting square of \eqref{eqn:G_Q_proof_setup} to get
	\begin{align*}
		\pi_{(G_2,G_2)} \circ \left(\pi_{(S_2,G_2)} \circ \theta_* \circ i_{S_1}\right)_*(q)^{-1} = \pi_{(G_2,G_2)} \circ \left(\pi_{(S_2,G_2)} \circ i_{S_2} \circ \theta \right)_*(q)^{-1}.
	\end{align*}

	This is the inverse of the left factor.
\end{proof}

\begin{remark}
	If we set $Q = S_1 \subseteq F_{S_1}$ (the minimum subset allowed), then  $G^Q$ is  $F_{S_1}$.
	In this case, $f$ always exists (and is  $\theta$) and \cref{thm:G_Q_homomorphism} tells us the standard theorem about homomorphisms from the free group.
\end{remark}

\begin{remark}
	If we set $Q = F_{S_1}$, then $G^Q$ is  $G$, $f$ is $g$ and \cref{thm:G_Q_homomorphism} tells us nothing more than  \cref{thm:homomorphism_squares}.
\end{remark}

We see that if $\theta \colon S_1 \to S_2$ is surjective, then the homomorphism $h \colon (G_1)^Q \to G_2$ resulting from \cref{thm:G_Q_homomorphism} is surjective.

Observing the form of \cref{def:G_Q}, it is clear that $\pi_{(S,G)}(S)$ is a generating set for $G^Q$, that is to say that $\pi_{(\overline{Q},G^Q)} \circ \left(\pi_{(S,G)} \circ i_S \right)_*$ is surjective.
If we set $Q=F_S$, then $G^Q \cong G$, so $G$ can be realised as a quotient of $G^Q$ via a surjection $p_Q \colon G^Q \to G$ in a way such that the following diagram commutes.

\begin{equation*}
	\begin{tikzcd}
		& F_S \ar[dl, "\pi_{\left(S,G^Q\right)}"'] \ar[dr, "\pi_{\left(S,G\right)}"] &
		\\ G^Q \ar[rr, "p_Q"]  & & G
	\end{tikzcd}
\end{equation*}

Furthermore, in the following diagram, it can be shown that the bottom map is injective.

\begin{equation}
	\begin{tikzcd}
		& Q \ar[dl, "\pi_{\left(S,G^Q\right)}"'] \ar[dr, "\pi_{\left(S,G\right)}"] &
		\\ \pi_{\left( S,G^Q \right)} (Q) \ar[rr, hook, "p_Q"]  & & \pi_{\left( S,G \right)}(Q)
	\end{tikzcd}
	\label{eqn:injectivity_of_P_Q_on_Q}
\end{equation}

\begin{lemma}
	\label{lem:alternative_presentation_for_G_Q}
	Given some group $G \cong \GroupPres{S}$ and $Q$ such that $S \subseteq Q \subseteq F_S$, we have that $G^Q$ is isomorphic to the following group by extending the natural identification of generators.
	\begin{equation}
		X \coloneqq \GroupPres{S \relations \Set{q_1 = q_2 \given q_1,q_2 \in Q, \; \pi_{(S,G) }(q_1) = \pi_{(S,G)}(q_2)}}
	\end{equation}
\end{lemma}

% \begin{proof}
% 	We begin by showing that $\id_S$ extends to a homomorphism $a \colon X \to G^Q$.
% 	Using \cref{thm:homomorphism_universal_prop}, it suffices to show $\pi_{\left(S,G^Q\right)}(q_1q_2^{-1}) = 1$ for all $q_1,q_2 \in Q$ such that $\pi_{(S,G)}(q_1) = \pi_{(S,G)}(q_2)$.
% 	Using \eqref{eqn:injectivity_of_P_Q_on_Q}, if $\pi_{(S,G)}(q_1) = \pi_{(S,G)}(q_2)$ then $\pi_{\left(S,G^Q\right)}(q_1) = \pi_{\left(S,G^Q\right)}(q_2)$.
% 	Then, since $\pi_{\left(S,G^Q\right)}$ is a homomorphism, we have
% 	\[
% 		\pi_{(S,G^Q)}\left(q_1q_2^{-1}\right) = \pi_{(S,G^Q)}(q_1)\pi_{(S,G^Q)}(q_2)^{-1} = 1
% 		.\]
%
% 	Now we work to show that $\id_S$ also extends to a homomorphism $b \colon G^Q \to X$.
% 	Using \cref{thm:G_Q_homomorphism}, it is sufficient to show that there exists a map $f$ to make the following diagram commute.
% 	\begin{equation*}
% 		\begin{tikzcd}
% 			& Q \ar[dl, "\pi_{(S,G)}"'] \ar[dr, "\pi_{(S,X)}"] &
% 			\\ G \ar[rr, "f"] & & X
% 		\end{tikzcd}
% 	\end{equation*}
% 	Such an $f$ exists if for all $q_1,q_2 \in Q$ such that $\pi_{\left( S,G \right)} (q_1) = \pi_{\left( S,G \right)} (q_2)$, we also have $\pi_{\left( S,X \right) }(q_1) = \pi_{\left( S,X \right) }(q_2)$.
% 	This is immediately true on inspection of the defining presentation for $X$.
%
% 	Thus, we have two homomorphisms, $a \colon X \to G^Q$, and $b \colon G^Q \to X$.
% 	Furthermore, by construction $\left( a \circ b \right) |_S = \id_S$, thus $a \circ b$ is the identity and $a$ is an isomorphism.
% \end{proof}
%
We could explore this construction a bit further, but I will return to the context of Artin and dual Artin groups.
In summary, we have the following.


Given a Coxeter system $\Gamma$, with generating set $S$, let $\overline{w}$ be some product of  all the elements of $S$ in  $F_S$, i.e.~$\pi_{(S,W_\Gamma)}(\overline{w})$ is a Coxeter element in $W_\Gamma$.
We call such a $\overline{w}$ a \emph{pre-Coxeter element}.
We can consider reflections in $F_S$ as all conjugates of  $S$, we denote this by  $\overline{R}$.
We can also act on $\overline{R}$-factorisations of $\overline{w}$ by the Hurwitz action all within $F_S$.
Doing so, we get a subset $\overline{R}$ that appear in minimal $\overline{R}$-factorisations of $\overline{w}$, we denote this set by $\minfact_{\overline{R}}(\overline{w})$.

Using \cref{lem:alternative_presentation_for_G_Q}, we have the following.

\begin{lemma}
	Given a Coxeter system $\Gamma$ generated by  $S$ and some pre-Coxeter element $\overline{w} \in F_S$.
	We have
	\[
		A^\vee_\Gamma \cong W_\Gamma^Q
		,\] where  $Q=\minfact_{\overline{R}}(\overline{w})$, obtained by application of the Hurwitz action in  $F_S$.
\end{lemma}

Then, by using \cref{thm:G_Q_homomorphism}, we have the following.

\begin{proposition}
	Suppose we have a Coxeter system $\Gamma$ with generating set $S$ and some pre-Coxeter element $\overline{w} \in F_S$.
	Let $Q = \minfact_{\overline{R}}(\overline{w})$.
	If in the following diagram, the bottom map q (a restriction of the standard projection from $A_\Gamma$ to  $W_\Gamma$) is injective, then $A^\vee_\Gamma \cong A_\Gamma$.
	\[
		\begin{tikzcd}
			& Q \ar[dl, "\pi_{(S,A_\Gamma)}"'] \ar[dr, "\pi_{(S,W_\Gamma)}"] & \\
			\pi_{(S,A_\Gamma)}(Q) \ar[rr, "q"] & & W_\Gamma
		\end{tikzcd}
	\]
\end{proposition}

This presents a new way of tackling the dual Artin group isomorphism problem, and suggests further study of $\minfact_{\overline{R}}(\overline{w}) \subseteq F_S$.
