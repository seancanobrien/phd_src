\subsection{Literature review}
Let $(W,S)$ denote a Coxeter system where $W$ is the Coxeter group and $S$ is its generating reflections.
Associated to this system is an edge labelled graph $\Gamma$.
The vertices of  $\Gamma$ correspond to elements of  $S$.
There is an edge labelled $m$ connecting the vertices corresponding to $s,t\in S$ if there is a relation between $s$ and  $t$ in  $W$ of the form  $(st)^m=1$.
We use $W_\Gamma$ and  $A_\Gamma$  as a shorthand for the Coxeter and Artin groups associated to a Coxeter system with graph $\Gamma$.
Note that $\Gamma$ also defines a Coxeter system, so we may use this notation in place of $(W,S)$.

Let $\Abs{\Gamma}$ denote the number of vertices in $\Gamma$.
Associated to a Coxeter system $\Gamma$, there is the so-called \emph{Tits cone}, denoted $T$, which is a subset of $\R^\Abs{\Gamma}$.
The Tits cone sees a canonical way of realising $W_\Gamma$ as a linear group.
The action of $S \subseteq W_\Gamma$ on  $T$ is by reflections (corresponding to a possibly non-standard inner product) through hyperplanes intersecting $T$.
All conjugates of $S$ in $W_\Gamma$ similarly act by reflections, and all such elements define a hyperplane intersecting $T$.
Let $H$ denote this set of hyperplanes.
We have that $H$ separates  $T$ into regions which are simplicial cones and are fundamental domains for the action of  $W_\Gamma$.

With this picture in mind, we define the complexified hyperplane arrangement $\overline{Y}$.

\[
	\overline{Y} \coloneq \left(T \times T\right) \setminus \bigcup_{h \in G} h \times h
	.\]

The action of $W_\Gamma$ on $T$ preserves pointwise the union of $H$, so we have an action of $W_\Gamma$ on $\overline{Y}$ and can consider the quotient space  $Y \coloneq W_\Gamma \backslash \overline{Y}$.
It is known by \cite{lek_homotopy_1983} that the fundamental group of $Y$ is the corresponding Artin group  $A_\Gamma$.
The $K(\pi,1)$ conjecture, attributed to Arnold, Brieskorn, Pham, and Thom, states that $Y$ always has contractible universal cover, that is, $Y$ is a $K(A_\Gamma,1)$ space.
This was known in the case where $A_\Gamma$ was the braid group by work of Fox and Neuwirth in \cite{fox_neuwirth_braid_1962} and proven in greater generality for all finite $W_\Gamma$ by Deligne in \cite{deligne_immeubles_1972}.
The conjecture was proven for certain other classes of $\Gamma$, including Large type and RAAGs in \cite{hendriks_hyperplane_1985} and \cite{charney_davis_finite_2016} respectively.

In 2021 Paolini and Salvetti proved the conjecture in the case where $W_\Gamma$ is affine \cite{paolini_salvetti_kpi1_2021}, which constituted significant progress on the problem.
Central to their proof was the so-called \emph{dual Artin group}, which we will denote $A^\vee_\Gamma$ and define shortly.
The construction of $A^\vee_\Gamma$ is due to Bessis in \cite{bessis_dual_2003}, which contextualises an alternative presentation of the braid group first developed by Birman, Ko and Lee in \cite{birman_etal_new_1998}.

Given a Coxeter system $\Gamma$ with generators $S$, we define $R$ to be all conjugates of  $S$ in  $W_\Gamma$.
We can consider $R$ as a generating set for $W_\Gamma$ and denote the corresponding Cayley graph  $\cay(W_\Gamma,R)$.
A \emph{Coxeter element} of  $W_\Gamma$ is any product of all the elements of $S$ (each occurring exactly once) in any order.
Fixing some Coxeter element $w$, we define $C_{w,R}$ to be the complete subgraph of $\cay(W_\Gamma,R)$ consisting of all geodesics from the identity to $w$.

\begin{definition}
	Given a Coxeter system $\Gamma$ with reflection set  $R \subseteq W_\Gamma$ and Coxeter element $w \in W_\Gamma$, a dual Artin group is defined as follows
	\[
		A^\vee_\Gamma \coloneq \GroupPres{\Set{r \in R \given r \text{ is an edge in } C_{w,R}} \relations \text{loops in } C_{w,R}}
		.\]
\end{definition}

Note that I previously said \emph{the dual Artin group}, but in the definition said \emph{a dual Artin group}.
This is because it is not known in general if $A^\vee_\Gamma$ depends on our choice of Coxeter element, though it is conjectured to not depend on that choice.
It is known that any Coxeter element has word length $\Abs{\Gamma}$ in $R$.
Thus, the dual Artin group construction depends on the set of length $\Abs{\Gamma}$ $R$--factorisations of  a Coxeter element.
In fact, it turns out to only depend on which $r \in R$ appear in such factorisations, as we will show.

By \cite{bessis_dual_2003, brady_watt_kp_2002} it is known that $A^\vee_\Gamma \cong A_\Gamma$ where  $W_\Gamma$ is finite.
The corresponding result for affine $W_\Gamma$ was first proved in \cite{mccammond_sulway_artin_2017} and also proved in \cite{paolini_salvetti_kpi1_2021}.
Recently, in \cite{resteghini_free_2024}, Resteghini proved that if $\Gamma_1$ and  $\Gamma_2$ satisfy $A^\vee_{\Gamma_1} \cong A_{\Gamma_1}$ and $A^\vee_{\Gamma_2}\cong A_{\Gamma_2}$, then the free product also satisfies this isomorphism problem, i.e.~$A^\vee_{\Gamma_1} \ast A^\vee_{\Gamma_2} \cong A_{\Gamma_1} \ast A_{\Gamma_2}$.
This free product also corresponds to a Coxeter system, which can be constructed by taking the disjoint union of $\Gamma_1$ and  $\Gamma_2$ and joining the two graphs by edges labelled by $\infty$.
It is shown in \cite{delucchi_etal_dual_2022}, that all infinite triangle groups (which happen to all be hyperbolic, in the Coxeter group sense) also satisfy this isomorphism problem.
Apart from these results mentioned, there are no other classes of $\Gamma$ where it is known that $A^\vee_\Gamma \cong A_\Gamma$.

The importance of the dual Artin group construction in recent proofs solidifies its relevance as an object of study.
The general lack of results for very obvious questions adds to its intrigue.
