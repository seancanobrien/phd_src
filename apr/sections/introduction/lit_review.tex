\subsection{Literature review}
Let $(W,S)$ denote a Coxeter system where $W$ is the Coxeter group and $S$ is its generating reflections.
Associated to this system is an edge labelled graph $\Gamma$.
The vertices of  $\Gamma$ correspond to elements of  $S$.
There is an edge labelled $m$ connecting the vertices corresponding to $s,t\in S$ if there is a relation between $s$ and  $t$ in  $W$ of the form  $(st)^m=1$.
We use $W_\Gamma$ and  $A_\Gamma$  as a shorthand for the Coxeter and Artin groups associated to a Coxeter system with graph $\Gamma$.
Note that $\Gamma$ also defines a Coxeter system, so we will use this notation in place of $(W,S)$.

Let $\Abs{\Gamma}$ denote the number of vertices in $\Gamma$.
Associated to a Coxeter system $\Gamma$, there is the so-called \emph{Tits cone}, denoted $T$, which is a subset of $\R^\Abs{\Gamma}$.
The Tits cone sees a canonical way of realising $W_\Gamma$ as a linear group.
The action of $W_\Gamma$ on  $T$ is by reflections (corresponding to a possibly non-standard inner product) through hyperplanes intersecting $T$.
These hyperplanes define a tiling of  $T$ where each tile is a simplical cone.
If we take
There is a conical, simplicial tiling of the Tits cone where each cell  cell corresponds to a fundamental domain of  $W_\Gamma$ acting on  $\R^\Abs{\Gamma}$.

