%! TEX root = ../../main.tex
For a set of symbols $\mathcal{S}$, define  $\mathcal{S}^*$ to be the language of all words in  $\mathcal{S}$.
Consider the following and compare with \cref{lem:extend_map_to_homomorphism}.

\begin{lemma}
	Let $G$ and $G^\prime$ be groups generated by $S$ and $S^\prime$ respectively.
	Let $\pi_G \colon (S \cup S^{-1})^* \to G$ denote multiplication of words in the group, and similarly for $\pi_{G^\prime}$.
	Suppose we have a map  $\theta \colon S \to S^\prime$.
	We can extend $\theta$ to a function $\theta^* \colon (S \cup S^{-1})^* \to (S^\prime \cup (S^\prime)^{-1})^*$ in the obvious way.
	Consider the following diagram.
	\[
		\begin{tikzcd}
			S \ar[d, "\theta"] \ar[r, hook] & (S \cup S^{-1})^* \ar[d, "\theta^*"] \ar[r, "\pi_G"] & G \\
			S^\prime \ar[r, hook] & (S^\prime \cup (S^\prime)^{-1})^* \ar[r, "\pi_{G^\prime}"] & G^\prime
		\end{tikzcd}
		.\]
	The left square commutes.
	If there exists a \emph{map} $f \colon G \to G^\prime$ that makes the right square commute, then $f$ is a homomorphism.
	\label{lem:homomorphism_squares}
\end{lemma}
\begin{remark}
	Note that if $G \cong \GroupPres{S \relations R}$ then such a map (and thus homomorphism) $f$ exists exactly when  $\pi_{G^\prime} \circ \theta^* (R) = 1$.
	The above lemma frames up the construction in \cref{lem:extend_map_to_homomorphism}, but does not provide any useful way to detect when such a homomorphism is possible.
	\label{rem:when_does_map_exist}
\end{remark}

\begin{definition}[Group with relations visibile in Q]
	Let $G \cong \GroupPres{S \relations R}$.
	Let  $Q$ be some language in  $S \cup S^{-1}$ such that, considering  $S$ as one-letter words, $S \subseteq Q$.
	We define the group with relations visible in $Q$,  $G^Q$ to be
	\[
		G^Q \coloneqq \GroupPres{\pi_G(Q) \relations \Set{\pi_G(q_1q_2\cdots q_n) = \pi_G(q_1)\pi_G(q_2) \cdots \pi_G(q_n) \given q_1\cdots q_n \in Q} }
		.\]
	Note that the generators $\pi_G(Q)$ are abstract generators.
	They do not necessarily inherit anything from $G$.
	The only equations that are necessarily true in  $G^Q$ (which are also tautologically true in $G$) are those spelled out by the language $Q$.
	\label{def:group_relations_visible_in_Q}
\end{definition}

\begin{remark}
	$G^Q$ is generated by the generators $\pi_G(S)$.
	$G$ is a quotient of  $G^Q$.
\end{remark}

\begin{theorem}
	Let $G \cong \GroupPres{S \relations R}$, let  $G^\prime$ be generated by $S^\prime$ and let $Q$ be a language in  $S \cup S^{-1}$ such that $S \subseteq Q$ as in \cref{def:group_relations_visible_in_Q}.
	Suppose we have a map $\theta \colon S \to S^\prime$ and define $\theta^*$ as in \cref{lem:homomorphism_squares}.
	For brevity, let $\overline{Q} \coloneqq \pi_G(Q)$.
	Consider the following diagram.
	\[
		\begin{tikzcd}
			S \ar[d, "\theta"] \ar[r, hook] & Q \ar[d, "\theta^*"] \ar[r, "\pi_G"] & \overline{Q} \ar[d, color=red, "\exists f ?"] \ar[r, hook] & (\overline{Q} \cup \overline{Q}^{-1})^* \ar[d, "\pi_{G^Q}"]\\
			S^\prime \ar[r, hook] & (S^\prime \cup (S^\prime)^{-1})^* \ar[r, "\pi_{G^\prime}"] & G^\prime & G^Q \ar[l, color=red, "\implies \exists g"']
		\end{tikzcd}
		.\]
	The left square commutes.

	Finally, the theorem is as follows.
	If there exists a map $f \colon \pi_G(Q) \to G^\prime$ that makes the middle square commute, then there is a homomorphism $g \colon G^Q \to G^\prime$ such that the right square commutes.
	\label{thm:commuting_diagram_GQ}
\end{theorem}
\begin{proof}
	Suppose we have such a diagram and map $f$.
	Note that, in $f$, we have a map from the generating set of $G^Q$ to $G^\prime$, which can be considered a generating set for $G^\prime$.
	Accordingly, replace $\theta$ with  $f$ in \cref{lem:homomorphism_squares}.
	\[
		\begin{tikzcd}
			\overline{Q} \ar[d, "f"] \ar[r, hook] & (\overline{Q} \cup \overline{Q}^{-1})^* \ar[d, "f^*"] \ar[r, "\pi_{G^Q}"] & G^Q \\
			G^\prime \ar[r, hook] & (G^\prime \cup (G^\prime)^{-1})^* \ar[r, "\pi_{G^\prime}"] & G^\prime
		\end{tikzcd}
		.\]
	If we can show we have a map $h \colon G^Q \to G^\prime$ such that the right square commutes, then by \cref{lem:homomorphism_squares} this is a homomorphism.
	This will also satisfy the necessary commutation relations.

	We can use the defining presentation for $G^Q \cong \GroupPres{\pi_G(Q) \relations R}$ from \cref{def:group_relations_visible_in_Q} and \cref{rem:when_does_map_exist} to show that $g$ exists.
	Let $R$ be our relations from \cref{def:group_relations_visible_in_Q}.
	Showing $\pi_{G^\prime} \circ f^*\left( R \right) = 1 $ will finish our proof.

	Let $(q_1q_2\cdots q_n) = q \in Q$.
	This corresponds to the relation $\pi_G(q_1\cdots q_n) = \pi_G(q_1)\cdots \pi_G(q_n) $ in $G^Q$.
	Each of $\pi_G(q)$ and the  $\pi_G(q_i)$ are in  $\pi_G(Q)$, so  $f^*(\pi_G(q)) = f(\pi_G(q))$ and  $f^*(\pi_G(q_i)) = f(\pi_G(q_i))$.
	By our assumed commuting diagram,  $f(\pi_G(q)) = \pi_{G^\prime}(\theta^*(q))$, so we have the following.
	\begin{align*}
		f^*(\pi_G(q)) & = f(\pi_G(q))                                                    \\
		              & = \pi_{G^\prime}(\theta^*(q))                                    \\
		              & = \pi_{G^\prime}(\theta^*(q_1\cdots q_n))                        \\
		              & = \pi_{G^\prime}(\theta(q_1) \cdots \theta(q_n))                 \\
		              & = \pi_{G^\prime}(\theta(q_1)) \cdots \pi_{G^\prime}(\theta(q_n)) \\
		              & = f(\pi_G(q_1)) \cdots f(\pi_G(q_n))                             \\
		              & = f^*(\pi_G(q_1)) \cdots f^*(\pi_G(q_n))
	\end{align*}
	Thus also
	\begin{align*}
		 & \pi_{G^\prime}(f^*(\pi_G(q)\pi_G(q_n)^{-1}\pi_G(q_{n-1})^{-1} \cdots \pi_G(q_1)^{-1})) =                   \\
		 & \pi_{G^\prime}(f^*(\pi_G(q_1))\cdots f^*(\pi_G(q_n))(f^*(\pi_G(q_n)))^{-1}\cdots (f^*(\pi_G(q_1)))^{-1}) = \\
		 & \pi_{G^\prime}(1)  = 1                                                                                     \\
	\end{align*}
\end{proof}

\begin{remark}
	Again, we note that the last part of the proof had no dependence on $G^\prime$.
	Once we had the map $f$, we just chased the diagram and everything popped out.
	This is because of the very specific form of the relations in  $G^Q$.
	I believe that  $G^Q$ is defined by the diagram in \cref{thm:commuting_diagram_GQ}, but I don't know enough category theory to explore that.
\end{remark}
\begin{remark}
	Note that in \cref{thm:commuting_diagram_GQ}, c.f.~\cref{lem:homomorphism_squares} we replaced $(S \cup S^{-1})^*$ with $Q$.
\end{remark}
\begin{remark}
	If we set $Q = S$ (the minimum language allowed), then  $G^Q$ is  $F_S$, the free group generated by  $S$.
	Then  $f$ always exists (and is  $\theta$) and \cref{thm:commuting_diagram_GQ} tells us the standard theorem about homomorphisms from the free group.
\end{remark}
\begin{remark}
	If we set $Q = (S \cup S^{-1})^*$ (the maximum language), then $G^Q$ is  $G$, $f$ is  $g$ and \cref{thm:commuting_diagram_GQ} tells us nothing more than  \cref{lem:homomorphism_squares}.
\end{remark}


